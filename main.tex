\documentclass[10pt, conference, compsocconf]{IEEEtran}
%\documentclass[10pt, conference, hidelinks]{IEEEtran}
\bibliographystyle{IEEEtran}

% Recommended macros
%\usepackage{amssymb}
\usepackage{amsmath,amssymb}
%\usepackage[cmex10]{amsmath}
\usepackage{array}
\usepackage{mdwtab}
\usepackage[caption=false,font=footnotesize]{subfig}
\usepackage{fixltx2e}
%\usepackage{stfloats}
\usepackage{dblfloatfix} 
\usepackage{url}
\usepackage[pdfusetitle, pdfauthor={Roland Glantz, Maria Predari and Henning Meyerhenke}]{hyperref}

% correct bad hyphenation here
\hyphenation{op-tical net-works semi-conduc-tor}

% Eigene Makros
\usepackage[dvipsnames]{xcolor}
\usepackage{xspace}
\usepackage{numprint}
\usepackage{algorithm}
\usepackage{algorithmicx}
\usepackage{algpseudocode}
\usepackage{float}
\usepackage{placeins}
\newfloat{algorithm}{t}{lop}
\usepackage{graphicx}

% equalize columns on last page
\usepackage{flushend}


\newtheorem{lemma}{Lemma}%[section]
\newtheorem{notation}{Notation}%[section]
\newtheorem{definition}{Definition}%[section]
\newtheorem{proposition}{Proposition}%[section]

\renewcommand{\floatpagefraction}{0.9}
%\newcommand{\vars}{\texttt}

%\newtheorem{consDefinition}[example]{Definition}
%\newtheorem{consTheorem}[example]{Theorem}
%\newtheorem{consProposition}[example]{Proposition}

% user-specified commands
%\newcommand{\R}{\mathbb{R}}
\newcommand{\N}{\mathbb{N}}
%\newcommand{\llminimal}{${\|\cdot\|}_2$-minimal\xspace}
\newcommand{\Pro}[1]{\mathbf{Pr} \left[\,#1\,\right]}
\newcommand{\pro}[1]{\mathbf{Pr} [\,#1\,]}
\newcommand{\Ex}[1]{\mathbb{E} \left[\,#1\,\right]}
\newcommand{\ex}[1]{\mathbb{E} [\,#1\,]}
\newcommand{\hit}{- [\pi]_s (H[v,s]-H[u,s])}

\newcommand{\centre}[1]{z_{#1}}
\newcommand{\centres}{Z}
\newcommand{\degree}{\operatorname{deg}}
\newcommand{\maxdeg}{\operatorname{maxdeg}}
\newcommand{\diam}{\operatorname{diam}}
\newcommand{\res}{\operatorname{res}}
\newcommand{\Res}{\operatorname{Res}}
\newcommand{\cond}{\operatorname{cond}}
\newcommand{\Cond}{\operatorname{Cond}}
\newcommand{\proxy}{\operatorname{proxy}}
\newcommand{\excond}{\operatorname{ex\_cond}}
\newcommand{\exres}{\operatorname{ex\_res}}
\newcommand{\exalg}{\operatorname{ex\_alg}}
\newcommand{\dist}{\operatorname{dist}}
\newcommand{\ord}{\operatorname{ord}}
\newcommand{\Vor}{\operatorname{Vor}}
\newcommand{\M}{\mathbf{M}}
\newcommand{\LL}{\mathbf{L}}
%\newcommand{\L}{\mathbf{L}}
\newcommand{\RR}{\mathbb{R}}
\newcommand{\f}{\hat{f}}
\newcommand{\e}{\mathbf{e}}
\newcommand{\dhat}{d}
\newcommand{\llminimal}{${\|\cdot\|}_2$-minimal\xspace}
\newcommand{\NP}{$\mathcal{NP}$}
\newcommand{\disjbigcup}{\mathop{\dot{\bigcup}}} 
\newcommand{\disjcup}{\mathop{\dot{\cup}}} 
\newcommand{\bigO}{\mathcal{O}} 
\newcommand{\polylog}{\operatorname{polylog}} 
\newcommand{\dotcup}{\stackrel{.}{\cup}}

\DeclareMathOperator{\intraWeight}{\it intraWeight}
\DeclareMathOperator{\interWeight}{\it interWeight}
\DeclareMathOperator{\subtreeVol}{\it subtreeVol}
\DeclareMathOperator{\cutWeight}{\it cutWeight}
\DeclareMathOperator{\conduct}{\it conduct}
\DeclareMathOperator{\inCutSet}{\it inCutSet}

\newcommand{\iec}{\textit{i.\,e.},\xspace}
\newcommand{\ie}{\textit{i.\,e.}\xspace}
\newcommand{\Ie}{\textit{I.\,e.}\xspace}
\newcommand{\egc}{\textit{e.\,g.},\xspace}
\newcommand{\eg}{\textit{e.\,g.}\xspace}
\newcommand{\Eg}{\textit{E.\,g.}\xspace}
\newcommand{\etal}{\textit{et al.}\xspace}
\newcommand{\Wlog}{w.\,l.\,o.\,g.\xspace}
\newcommand{\wrt}{w.\,r.\,t.\xspace}
\newcommand{\cf}{cf.\xspace}

\newcommand{\mswap}{\textsc{TiME}\xspace}
\newcommand{\karma}{\textsc{KarMa}\xspace}
\newcommand{\greedyallc}{\textsc{GreedyAllC}\xspace}
\newcommand{\dibap}{\textsc{DibaP}\xspace}
\newcommand{\pdibap}{\textsc{PDibaP}\xspace}
\newcommand{\bubble}{\textsc{Bubble}\xspace}
\newcommand{\smooth}{\textsc{Smooth}}
\newcommand{\bubfosc}{\textsc{Bubble-FOS/C}\xspace}
\newcommand{\bubfost}{\textsc{Bubble-FOS/T}\xspace}
\newcommand{\metis}{\textsc{METIS}\xspace}
\newcommand{\kmetis}{\textsc{kMeTiS}\xspace}
\newcommand{\parmetis}{\textsc{ParMETIS}\xspace}
\newcommand{\kappart}{\textsc{KaPPa}\xspace}
\newcommand{\kahip}{\textsc{KaHIP}\xspace}
\newcommand{\jostle}{\textsc{Jostle}\xspace}
\newcommand{\zoltan}{\textsc{Zoltan}\xspace}
\newcommand{\parkway}{\textsc{Parkway}\xspace}
\newcommand{\graclus}{\textsc{Graclus}\xspace}
\newcommand{\party}{\textsc{Party}\xspace}
\newcommand{\scotch}{\textsc{Scotch}\xspace}
\newcommand{\thrsh}{$\mathtt{thrsh}$\xspace}
\newcommand{\trunccons}{\textsc{TruncCons}\xspace}
\newcommand{\consol}{\texttt{Consolidation}\xspace}
\newcommand{\consols}{\texttt{Consolidations}\xspace}
\newcommand{\asspart}{\texttt{AssignPartition}\xspace}
\newcommand{\assclus}{\texttt{AssignCluster}\xspace}
\newcommand{\asssd}{\texttt{AssignSubdomain}\xspace}
\newcommand{\compcen}{\texttt{ComputeCenters}\xspace}
\newcommand{\initcen}{\textsc{LoadBasedInitialCenters}\xspace}
\newcommand{\NN}{\mbox{\rm I$\!$N}}
\newcommand{\closu}[1]{\overline{#1}}
\newcommand{\djoko}{Djokovi\'{c} relation\xspace}
\newcommand{\djokoRelated}{Djokovi\'{c}\xspace related\xspace}
\newcommand{\subE}[1]{{#1}_{\mathcal{E}}}
\newcommand{\comm}{\operatorname{Co}}

\newcommand{\argmin}{\operatorname{argmin}\xspace}
\newcommand{\argmax}{\operatorname{argmax}\xspace}
\newcommand{\cc}{\operatorname{Coco}\xspace}
\newcommand{\divers}{\operatorname{Div}\xspace}
\newcommand{\ccd}{\operatorname{Coco^+}\xspace}
\newcommand{\dil}{\operatorname{dil}\xspace}
\newcommand{\contract}{\operatorname{contract}\xspace}
\newcommand{\assemble}{\operatorname{assemble}\xspace}
\newcommand{\modulo}{\operatorname{mod}\xspace}

\newcommand{\parent}{\operatorname{parent}\xspace}
\newcommand{\oldParent}{\operatorname{oldParent}\xspace}
\newcommand{\newParent}{\operatorname{newParent}\xspace}
\newcommand{\newParentLabel}{\operatorname{newParentLabel}\xspace}
\newcommand{\prefLabel}{\operatorname{prefLabel}\xspace}

\newcommand{\initial}{\textsc{Identity}\xspace}
\newcommand{\initialM}{\textsc{Identity}_{\textsc{g}} \xspace}
\newcommand{\initialMO}{\textsc{Identity}_{\textsc{n}} \xspace}
\newcommand{\random}{\textsc{Random}\xspace}
\newcommand{\randomM}{\textsc{Random}_{\textsc{g}} \xspace}
\newcommand{\randomMO}{\textsc{Random}_{\textsc{n}} \xspace}
\newcommand{\rcm}{\textsc{RCM}\xspace}
\newcommand{\rcmM}{\textsc{RCM}_{\textsc{g}} \xspace}
\newcommand{\rcmMO}{\textsc{RCM}_{\textsc{n}} \xspace}
\newcommand{\durebi}{\textsc{DRB}\xspace}
\newcommand{\durebiM}{\textsc{DRB}_{\textsc{g}} \xspace}
\newcommand{\durebiMO}{\textsc{DRB}_{\textsc{n}} \xspace}
\newcommand{\greedyall}{\textsc{GreedyAll}\xspace}
\newcommand{\greedyallM}{\textsc{GreedyAll}_{\textsc{g}} \xspace}
\newcommand{\greedyallMO}{\textsc{GreedyAll}_{\textsc{n}} \xspace}
\newcommand{\greedyallcM}{\textsc{GreedyAllC} \xspace}
\newcommand{\greedyallcMO}{\textsc{greedyAllC}_{\textsc{n}} \xspace}
\newcommand{\greedymin}{\textsc{GreedyMin}\xspace}
\newcommand{\greedyminM}{\textsc{GreedyMin}_{\textsc{g}} \xspace}
\newcommand{\greedyminMO}{\textsc{GreedyMin}_{\textsc{n}} \xspace}
\newcommand{\greedyminc}{\textsc{GreedyMinC}\xspace}
\newcommand{\greedymincM}{\textsc{GreedyMinC}_{\textsc{g}} \xspace}
\newcommand{\greedymincMO}{\textsc{GreedyMinC}_{\textsc{n}} \xspace}
\newcommand{\walshawlarge}{\textsc{WalshawLarge}\xspace}
\newcommand{\complexnets}{\textsc{ComplexNets}\xspace}
\newcommand{\gpmetis}{\textsc{gpMetis}\xspace}
\newcommand{\ndmetis}{\textsc{ndMetis}\xspace}


\newcommand{\cfc}{$\mathcal{CFCC}$}



\renewcommand{\arraystretch}{1.4}

\newcommand{\mpre}[1]{\textcolor{red}{#1}\xspace}
\newcommand{\gsto}[1]{\textcolor{blue}{#1}\xspace}
% \renewcommand{\hmey}[1]{}
%\renewcommand{\rgla}[1]{}


\pagestyle{plain}


% %%%%%%%%%%%%%%%%%%%%%%%%%%%%%%%%%%%%%%%%%%%%%%%%
\author{
\IEEEauthorblockN{Blah Blah}
\IEEEauthorblockA{Humboldt University, Berlin, Germany\\
Email: kdfkfd@informatik.hu-berlin.de}
\and
\IEEEauthorblockN{Blah Blah}
\IEEEauthorblockA{Humboldt University, Berlin, Germany\\
Email: lrpwor@informatik.hu-berlin.de}
%% \IEEEauthorblockN{Gabriel Stoszek}
%% \IEEEauthorblockA{University of Cologne, Cologne, Germany\\
%% Email: gstoszek@gmail.com}
%% \and
%% \IEEEauthorblockN{Maria Predari}
%% \IEEEauthorblockA{Humboldt University, Berlin, Germany\\
%% Email: maria.predari@informatik.hu-berlin.de}
%% \and
%% \IEEEauthorblockN{Henning Meyerhenke}
%% \IEEEauthorblockA{Humboldt University, Berlin, Germany\\
%% Email: henning.meyerhenke@informatik.hu-berlin.de}
%% \and
%% \IEEEauthorblockN{Eugenio Angriman}
%% \IEEEauthorblockA{Humboldt University, Berlin, Germany\\
%% Email: eugenio.angriman@informatik.hu-berlin.de}
}


\makeatletter
\algrenewcommand\ALG@beginalgorithmic{\small}
\makeatother

\begin{document}
 \title{Fast Algorithms for Group Current-flow Closeness Centrality}

\maketitle 

\begin{abstract}
  Current flow closeness centrality (\cfc) has a better discriminating ability than the
  ordinary closeness centrality based on shortest paths. \mpre{One line to define \cfc~for one node}
  Recently, the definition of the problem was extended to the group case~\cite{Li18}.
  More precisely, one aims at finding a subset $S$ of $k$ vertices to maximize
  its \cfc $C(S)$. In this work we propose two new algorithms, an exact and
  an approximation one. Both algorithms use the multilevel framework to reduce
  the graph into smaller dimensions and
  solve the underlying laplacian system on a smaller instance.
  The solution is refined back to the original input
  using local algorithms. %\mpre{Heuristics for the approx, closed formulas for the exact}
  The approximation algorithm employs fast techniques that exploit knownledge
  from previous levels to reduce the search space. Extensive experiments on synthetic
  and real networks demonstrate that our algorithms
  outperform existing work. As a result, our algorihtms are able to process massive
  networks with more than a billion vertices in less than \mpre{numbers}.
\end{abstract}

\begin{IEEEkeywords}
network analysis; current-flow closeness centrality
\end{IEEEkeywords}



%%%%%%%%%%%%%%%%%%%%%%%%%%%%%%%%%%%%%%%%%%%%%%%%%%
%
%
%%%%%%%%%%%%%%%%%%%%%%%%%%%%%%%%%%%%%%%%%%%%%%%%%%
\section{Introduction}
\label{sec:intro}
Networks are ubiquitous in so many areas of science. Some examples include.
Finding the most important nodes in the graph can be very beneficial for the application.
Many existing measurements of the importance of nodes exist. Examples are
closeness centrality~\cite{}, betweenness centrality~\cite{}, etc.
However the most common ones are based on simple shortest paths information
and may fail to reveal important properties of the network. Resiliance to
changes in the network should be taken into consideration. Nowadays many
applications include data that change dynamically over time. Example: social networks
are the addition/deletion of friendships. In suc dynamic scenarios resiliance
of the node become even more important...

\mpre{Goup closeness: add why we want to look at the goup case. }

Unfortunately, the computation of the \cfc solely for one node is expensive.
\mpre{Explain more here, see Elisabetta's paper on the connection to the laplacian solution~\cite{Bergamini16}}
Computing the \cfc for all nodes in the graph becames prohibitive even for smaller graphs.
in this paper we propose fast algorithms for the computation of the group current-flow
closeness centrality in order to be able to process massive graphs.


\paragraph{Related work and motivation}
\mpre{Related work should not have a seperate section since it is rather small and recent.}
Recently, the problem of group current-flow closeness centrality was introduced~\cite{Li18}.
The authors formally define the problem and prove theoretical properties such as submodularity existence.
They also present two algorithms, a deterministic one and an approximation one that
are extenstions of previous work done in the context of group closeness centrality~\cite{Bergamini18}.


%%%%%%%%%%%%%%%%%%%%%%%%%%%%%%%%%%%%%%%%%%%%%%%%%%
\paragraph{Problem defintion}
\mpre{define graph, the problem, the connection to laplacian}
%%%%%%%%%%%%%%%%%%%%%%%%%%%%%%%%%%%%%%%%%%%%%%%%%%
\paragraph{Contribution and outline}

\section{Exact algorithm}
\label{sec:multiSwap}
\subsection{Running time analysis}

\section{Approximation algorithm}
\label{sec:multiSwap}
\subsection{Running time analysis}

%%%%%%%%%%%%%%%%%%%%%%%%%%%%%%%%%%%%%%%%%%%%%%%%%%
%
%
%%%%%%%%%%%%%%%%%%%%%%%%%%%%%%%%%%%%%%%%%%%%%%%%%%
\section{Experiments}
\label{sec:experiments}

%%%%%%%%%%%%%%%%%%%%%%%%%%%%%%%%%%%%%%%%%%%%%%%%%%
\subsection{Description of experiments}
\label{subsec:setup}
%
In this section we specify our test instances, our experimental setup
and the way we evaluate the improvements induced by our methods.

The application graphs are the 15 complex networks used
in~\cite{Safro2012a} %
All computations are based on sequential C++ code and run on a workstation with two
4-core Intel(R) Core(TM) i7-2600K processors at 3.40GHz.
% Our code is written
%in C++ and compiled with GCC 4.7.1.


\begin{table*}[!b]
\caption{Complex networks used for benchmarking.}
\label{tab:complex}
\begin{center}
%\begin{footnotesize}
%\scalebox{0.8}{
\begin{tabular}{ l | r | r | c }
Name & \#vertices & \#edges & Type\\ \hline \hline
p2p-Gnutella          & \numprint{6405}   & \numprint{29215}    & file-sharing network\\\hline
PGPgiantcompo         & \numprint{10680}  & \numprint{24316}    & largest connected component in network of PGP users\\\hline
email-EuAll           & \numprint{16805}  & \numprint{60260}    & network of connections via email\\\hline
as-22july06           & \numprint{22963}  & \numprint{48436}    & network of internet routers \\\hline
soc-Slashdot0902      & \numprint{28550}  & \numprint{379445}   & news network\\\hline
loc-brightkite\_edges & \numprint{56739}  & \numprint{212945}   & location-based friendship network\\\hline
loc-gowalla\_edges    & \numprint{196591} & \numprint{950327}   & location-based friendship network \\\hline
citationCiteseer      & \numprint{268495} & \numprint{1156647}  & citation network\\\hline
coAuthorsCiteseer     & \numprint{227320} & \numprint{814134}   & citation network\\\hline
wiki-Talk             & \numprint{232314} & \numprint{1458806}  & network of user interactions through edits\\\hline
coAuthorsDBLP         & \numprint{299067} & \numprint{977676}   & citation network\\\hline
web-Google            & \numprint{356648} & \numprint{2093324}  & hyperlink network of web pages\\\hline
coPapersCiteseer      & \numprint{434102} & \numprint{16036720} & citation network\\\hline
coPapersDBLP          & \numprint{540486} & \numprint{15245729} & citation network\\\hline
as-skitter            & \numprint{554930} & \numprint{5797663}  & network of internet service providers\\\hline
\end{tabular}
%}
\end{center}
\end{table*}


%%%%%%%%%%%%%%%%%%%%%%%%%%%%%%%%%%%%%%%%%%%%%%%%%%
\subsection{Experimental Results}
\label{subsec:results}
%
The detailed experimental results are displayed in Tables~\ref{tab:extra_social_initial_1}
through~\ref{tab:extra_social_brandfassL_2}. Here is our summary and interpretation:


%%%%%%%%%%%%%%%%%%%%%%%%%%%%%%%%%%%%%%%%%%%%%%%%%%
\section{Conclusions}
\label{sec:conclusions}

\section*{Acknowledgments}
%
This work is partially supported by German Research Foundation (DFG) grant ME 3619/2-1.
Large parts of this work were carried out while H.M. was affiliated with Karls\-ruhe
Institute of Technology.


%\hmey{bibitems are too long if they contain links.}

% \clearpage 
%%
%% Bibliography
%%
%\nocite{*}
\bibstyle{IEEEtran}
\bibliography{biblio}                                     

\end{document}
